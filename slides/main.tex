\documentclass{beamer}

\usepackage[utf8]{inputenc} 
\usepackage[T1]{fontenc}
\usepackage{amsmath}
\usepackage{tikz}
\usepackage{listings}
\usetikzlibrary{shapes.arrows, fadings}

\usetheme{Madrid}
\usefonttheme[onlymath]{serif}

\begin{document}

\title{Formal languages and compilers}
\subtitle{Projects 2018}
\author{Henri LEFEBVRE}

\maketitle

\begin{frame}\tableofcontents\end{frame}

\section{Top-down parser : begin...end}
\begin{frame}[fragile]
    \frametitle{Introduction}

    \begin{block}{Problem statement}
        Parse input strings $s$ respecting the following rules :
        \begin{itemize}
            \item $s$ must start with "begin" and end with "end"
            \item both "begin...end" and "\{...\}" define block structures which must not overlap in $s$
        \end{itemize}
    \end{block}

    \begin{minipage}{.45\textwidth}
        \begin{exampleblock}{Valid example}
            \begin{lstlisting}
begin
    <text>
    {
        <text>
    }
    <text>
end
            \end{lstlisting}
        \end{exampleblock}
    \end{minipage}
    \hspace{.2cm}
    \begin{minipage}{.45\textwidth}
        \begin{alertblock}{Invalid example}
            \begin{lstlisting}
begin
    <text>
    {
        begin <text>
    }
    end <text>
end
            \end{lstlisting}
        \end{alertblock}
    \end{minipage}

\end{frame}

\begin{frame}
    \frametitle{LL Grammar}
    
    \textbf{Grammar definition :}
    \begin{align*}
        <pgm> &::= <stmts> \$\\
        <stmts> &::= \textrm{begin} <txt> \textrm{end}\\
        <txt> &::= <letter><txt> | <stmts><txt> | \{ <txt> \} | \epsilon\\
        <letter> &::= a|...|z
    \end{align*}

    \textbf{Predicitve parsing :}
    {\tiny
        \begin{center}
            \begin{tabular}{|c|c|c|c|c|c|c|c|}
                \hline
                NT symbol & [a-z] & \{ & \} & begin & end & \$ & $\epsilon$ \\\hline
                $<pgm>$ & & & & $<stmts>$ & & & \\\hline
                $<stmts>$ & & & & $\textrm{begin}<txt>\textrm{end}$ & & & \\\hline
                $<txt>$ & $<letter><txt>$ & $\{<txt>\}$ & $\epsilon$ & $<stmts><txt>$ & $\epsilon$ & & \\\hline
                $<letter>$ & $\epsilon$ & & & & & & \\\hline
            \end{tabular}
        \end{center}
    }
\end{frame}

\begin{frame}
    \frametitle{Implementation}
    \begin{itemize}
        \item Implmented in C++
        \begin{itemize}
            \item $Token$ class to represent tokens
            \item $ParsingTable$ class to link token and their predicted successors
            \item $Buffer$ class to manage efficient reading of input string
            \item $Interpreter$ class to make actions when rules are recognized
            \item $SyntacticAnalyser$ class to generate token from the $Buffer$ class
        \end{itemize}
        \item see https://github.com/hlefebvr/unige-top-down-parser
    \end{itemize}
\end{frame}

\begin{frame}\tableofcontents\end{frame}

\section{Bottom-up parser : NLP with multiple clauses}

\begin{frame}
    \frametitle{Introduction}
    \begin{block}{Problem statement}
        \begin{itemize}
            \item Parse in-english input strings
            \item Decompose multiple-clause sentences into several single-clause sentences
        \end{itemize}
    \end{block}

    \begin{exampleblock}{Worked example}
        The boy [who is tall, skinny and strange] smiles and laughs.
        \begin{itemize}
            \item The boy is tall
            \item The boy is skinny
            \item The boy is strange
            \item The boy smiles
            \item The boy laughs
        \end{itemize}
    \end{exampleblock}
\end{frame}

\begin{frame}
    \frametitle{LALR Grammar}

    {\tiny
    \begin{align*}
        <whole\_sentence> &::= <sentence> .\\
        <sentence> &::= <np?> <vp>\\
        <np> &::= <ng> <rel\_clause?>\\
        <ng> &::= DET?\; CARD?\; ORD?\; QUANT?\; <ap>? NOUN\\
            & | <np> COORD <np>\\
        <rel\_clause> &::= REL\_WORD <sentence>\\
        <vp> &::= <single\_vp> | <add\_vp>\\
        <single\_vp> &::= <conjugated\_verb> <np?>\\
            & | <conjugated\_verb> <ap?>\\
        <add\_vp> &::= <vp> COORD <vp>\\
        <conjugated\_verb> &::= NEGATION?\; CONJUGATED\_BE\\
            & | AUXILIARY\; NEGATION?\; VERB?\\
            & | VERB\\
        <ap> &::= ADJ | <ap> <ap> | <ap> COORD <ap>
    \end{align*}
    }

    \textbf{An external dictionary defines COORD, NOUN, ADJ, NEGATION, CONJUGATED\_BE, VERB, ... which are token returned by the lexer}
\end{frame}

\begin{frame}
    \frametitle{Implementation}
    \begin{itemize}
        \item Python 3.0
        \item Lark parsing library (https://github.com/lark-parser/lark)
        \begin{itemize}
            \item with a custom lexer making use of custom dictionary
        \end{itemize}
        \item Idea : transform the parsing tree into a simpler tree where, given one level of exploration (from bottom to up), we end up treating structures like $<np><rel\_clause?><vp>$
        \item see https://github.com/hlefebvr/nlp-multiple-clauses-to-single
    \end{itemize}
\end{frame}

\begin{frame}
    \frametitle{Results}
    \textbf{he can speak italian and english.}
    \begin{itemize}
        \item he can speak italian.
        \item he can speak english.
    \end{itemize}
\end{frame}

\begin{frame}
    \frametitle{Results}
    \textbf{the boy and the cat who is tall, skinny and strange smile and laugh.}
    \begin{itemize}
        \item the cat is tall.
        \item the cat is skinny.
        \item the cat is strange.
        \item the boy smile.
        \item the boy laugh.
        \item the cat smile.
        \item the cat laugh.
    \end{itemize}
\end{frame}

\begin{frame}
    \frametitle{Results}
    \textbf{the dog eat a man who is tall.}
    \begin{itemize}
        \item a man is tall.
        \item the dog eat a man.
    \end{itemize}
\end{frame}

\begin{frame}
    \frametitle{Results}
    \textbf{the boy whom I met yesterday cried and laughed.}
    \begin{itemize}
        \item I met yesterday the boy
        \item the boy cried.
        \item the boy laughed.
    \end{itemize}
\end{frame}

\begin{frame}
    \frametitle{Limitations}
    \begin{itemize}
        \item Using a context free grammar enforces us to make choices. \\ Example : "this" recognized as noun or determinant ?
        \begin{itemize}
            \item this dog is eating. ("this" as determinant)
            \item this is strange. ("this" as noun)
        \end{itemize}
    \end{itemize}
\end{frame}

\end{document}